%------------------------------------------------------------------------------

\chapter{Συμπεράσματα - Μελλοντική Εργασία}
\label{chapter6} 

Το παρόν εγχείρημα προέκυψε παραγωγικό. Πραγματοποιήσαμε και προσφέραμε στο ευρή
κοινό μια υλοποίηση της μεθόδου ανάλυσης μερικής διαφυγής και βελτιστοποίησης
μέσω αντικατάστασης βαθμωτών και συμβάλαμε φυσικά με τον δική μας μικρή βοήθεια
στην βελτίωση της ταχύτητας του μεταγλωττιστή του PyPy.

Βλέπουμε από το προηγούμενο κεφάλαιο τα νούμερα. (todo)

Ο μεγάλος όγκος των \texttt{malloc}s αφαιρούνται κατά την πρώτη \textit{build-
in} βελτιστοποίηση του PyPy (\textit{non-partial escape
analysis}\footnote{escape.py}). Η δικιά μας μέθοδος πυροδοτείται αργότερα και
παρόλα αυτά επιτυγχάνει να αποφείγει περισσότερα \texttt{getfield}s.

Καταλήξαμε ότι δεν μπορεί να γίνει ανάλυση και υλοποίηση της μεθόδου αυτής για
δυναμικές γλώσσες χωρίς να λάβουμε υπόψιν μας το aliasing και τις λεπτομέρεις
που επιφέρει. Φυσικά αυτό το πρόβλημα δεν υπάρχει στις στατικές γλώσσες καθώς ο
προγραμματιστής φροντίζει για τους τύπους, ενώ στην περίπτωση μας είναι
αρμοδιότητα του μεταγλωττιστής μας, οπότε το aliasing των τύπων υπεισερχεται
σχεδόν σε όλη την έκταση των προγραμμάτων.

%------------------------------------------------------------------------------