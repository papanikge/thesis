%----------------------------------------------------------------
%	PACKAGES AND OTHER DOCUMENT CONFIGURATIONS
%----------------------------------------------------------------

\documentclass[11pt, english, singlespacing, parskip, headsepline]{MastersDoctoralThesis}

\usepackage[utf8]{inputenc}
\usepackage[T1]{fontenc}
\usepackage{palatino} %%% FONT

% BIBLIOGRAPHY
\usepackage[backend=bibtex,natbib=true]{biblatex}
\addbibresource{graphy.bib} % bibliography filename
\usepackage[autostyle=true]{csquotes}

\usepackage{fontspec}
\usepackage{xunicode}
\usepackage{hyperref}
\usepackage{xltxtra}
\setromanfont{FreeSerif}
\setsansfont{FreeSans}
\setmonofont{FreeMono}
\usepackage{mathtools}
\usepackage{pdfpages}
\usepackage{enumerate}
\usepackage{listings}
\usepackage{ulem}
\usepackage{float}
\usepackage{calrsfs}
\usepackage{amssymb}
\usepackage[utf8]{inputenc}
\usepackage{amsmath}
\usepackage{graphicx}
\usepackage{amssymb,amsmath}
\usepackage{hyperref}
\usepackage{paralist}
\usepackage{url}
\usepackage{xgreek}
\usepackage{epigraph}
\usepackage{color}

%---------------------------------------------------------
%	MARGIN SETTINGS
%---------------------------------------------------------

\geometry{
	paper=a4paper, % Change to letterpaper for US letter
	inner=3cm, % Inner margin
	outer=3cm, % Outer margin (was 3.8)
	%bindingoffset=2cm, % Binding offset
	top=1.5cm, % Top margin
	bottom=1.5cm, % Bottom margin
	%showframe,% show how the type block is set on the page
}

% PARAGRAPHS (MINE == Maybe remove it)
% \setlength{\parskip}{\baselineskip}%
% \setlength{\parindent}{0pt}%

%---------------------------------------------------------
%	THESIS INFORMATION
%---------------------------------------------------------

\thesistitle{Μελέτη και Αξιοποίηση Τεχνικών Ανάλυσης Μερικής Διαφυγής και Αντικατάστασης Βαθμωτών για Στατική Βελτιστοποίσηση στον Μεταγλωττιστή pypy}
\supervisor{Ιωάννης \textsc{Γαροφαλάκης}}
\examiner{}
\degree{Engineer's degree}
\author{Γεώργιος \textsc{Παπανικολάου}}
\subject{Computer Science \& Informatics}
\university{\href{http://www.upatras.gr/el}{Πανεπιστήμιο Πατρών}}
\department{\href{https://www.ceid.upatras.gr/}{Τμήμα Μηχανικών Η/Υ \& Πληροφορικής}}
\faculty{}

\hypersetup{pdftitle=\ttitle} % pdf's title
\hypersetup{pdfauthor=\authorname} % author

%---------------------------------------------------------
%	COLOR (CODE) SETTINGS
%---------------------------------------------------------

\definecolor{mygreen}{rgb}{0,0.6,0}
\definecolor{mygray}{rgb}{0.5,0.5,0.5}
\definecolor{mymauve}{rgb}{0.58,0,0.82}

\lstset{ %
  backgroundcolor=\color{white},
  basicstyle=\footnotesize,
  breakatwhitespace=false,
  breaklines=true,
  captionpos=b,
  commentstyle=\color{mygreen},
  deletekeywords={...},
  escapeinside={\%*}{*)},
  extendedchars=true,
  frame=single,
  keepspaces=true,
  columns=flexible, %
  keywordstyle=\color{red},
  language=Python,
  rulecolor=\color{black},
  numbers=left,
  showspaces=false,
  showstringspaces=false,
  showtabs=false,
  stringstyle=\color{mymauve},
  tabsize=2,
  title=\lstname
}

%---------------------------------------------------------
%	TITLE PAGE
%---------------------------------------------------------

\begin{document}
\frontmatter
\pagestyle{plain}

\begin{titlepage}
\begin{center}

\begin{figure}[H]
\centering
\includegraphics[width=0.4\textwidth]{Upatras.jpg}
\end{figure}

\textsc{\Large Διπλωματική Εργασία}\\[0.5cm]

\HRule \\[0.4cm] % Horizontal line
{\huge \bfseries \ttitle}\\[0.4cm] 
\HRule \\[1.5cm] % Horizontal line
 
\begin{minipage}{0.4\textwidth}
\begin{flushleft} \large
\emph{Author:}\\
\href{http://www.github.com/papanikge}{\authorname}
\end{flushleft}
\end{minipage}
\begin{minipage}{0.4\textwidth}
\begin{flushright} \large
\emph{Supervisors:} \\
\href{http://athos.cti.gr/garofalakis/}{\supname}
\href{http://athos.cti.gr/nikolako/}{Αθανάσιος \textsc{Νικολακόπουλος}}
\end{flushright}
\end{minipage}\\[3cm]
 
{\large \today}\\[4cm]
 
\vfill
\end{center}
\end{titlepage}


%-----------------------------------------------------------
%	QUOTATION PAGE
%-----------------------------------------------------------
\newpage\null\thispagestyle{empty}\newpage
\vspace*{0.2\textheight}

\noindent\textit{Nanos Gigantum Humeris Insidentes}\bigbreak

\hfill – Bernard of Chartres

%-----------------------------------------------------------
%	ABSTRACT PAGE
%-----------------------------------------------------------

\begin{abstract}
\addchaptertocentry{Περίληψη}

Κύριως σκοπός αυτού του άρθρου είναι να συνοδεύει την απόπειρα σχεδίασης και υλοποίησης 

\end{abstract}

%---------------------------------------------------------------
% SECOND ENGLISH ABSTRACT (VERBATIM AND INLINED !!!)
\newpage\null\thispagestyle{empty}\newpage
\checktoopen
\tttypeout{\abstractname}
\null\vfil
\thispagestyle{plain}
\addchaptertocentry{English Abstract}
\begin{center}
{\normalsize \MakeUppercase{\href{http://www.upatras.gr/en}{University of Patras}} \par}
\bigskip
{\huge\textit{Abstract} \par}
\bigskip
{\normalsize \href{https://www.ceid.upatras.gr/en/}{Computer Engineer \& informatics Department} \par}
\bigskip
{\normalsize Engineer's Degree\par}
\bigskip
{\normalsize\bfseries Study and Implementation of Partial Escape Analysis and Scalar Replacement Methods for Static Optimization in the pypy Compiler Framework \par}
\medskip
{\normalsize Georgios Papanikolaou \par}
\bigskip
\end{center}

This document first and foremost is the companion of an attempt to improve the
escape analysis of the pypy interpreter. Although it also aims to shine light
at the peculiar craft of compiler optimization and will try to inform the
reader of the nuisances and problems that the engineers face throughout the
designing process. We will elaborate on general problems based on dynamic
language design, as well as problems that we experienced specifically with
Python and with the PyPy framework.  Furthermore we will expand on the details
of compiler optimization with intricate details on the static analysis of
graphs, partial escape analysis and scalar replacement. In addition we will
design and implement a backend optimization module for pypy. It will serve as
an example of the said theory, but we will also try to fully integrate it into
the whole pypy project, in order of course, to improve, as we already said, the
overall speed of the interpreter. Last but not least, we will benchmark our
implementation with the one that pypy is already using and with unoptimized
code.


%-----------------------------------------------------------
%	ACKNOWLEDGEMENTS
%-----------------------------------------------------------

\newpage\null\thispagestyle{empty}\newpage
\thispagestyle{plain}
\begin{center}{\huge\textit{Acknowledgements – Ευχαριστίες}\par}\end{center}
\vspace{3cm}

I'd like to thank Carl Friedrich Bolz for his valuable and eye-opening help.

%-----------------------------------------------------------

%% TOC:
\tableofcontents

%%%%%%%%%%%%%%%%%%%%%%%%%%%%%%%%%%%%%%%%%%%%%%%%%%%%%%%%%%%%
%	THESIS CONTENT - CHAPTERS
%%%%%%%%%%%%%%%%%%%%%%%%%%%%%%%%%%%%%%%%%%%%%%%%%%%%%%%%%%%%


\mainmatter % Begin numeric arithmetic

\pagestyle{thesis}

%------------------------------------------------------------------------------

\chapter{Εισαγωγή}
\label{chapter1} % \ref{}

\section{Γενικά}

Σε αυτό το κεφάλαιο αποσκοπούμε να ενημερώσουμε τον αναγνώστη γενικά περί
δυναμικών γλωσσών προγραμματισμού και πιο συγκεκριμένα για την διαδικασία
μεταγλώττισης τέτοιων γλωσσών και τα προβλήματα που αντιμετωπίζει κανείς.
Θεωρούμε ότι ο αναγνώστης ήδη κατέχει μια σχετικά καλή ιδέα για προγραμματισμό
για τις κάποιες από τις λεπτομέρειες που διέπουν τους μεταφραστές τους.

%------------------------------------------------------------------------------

\section{Δυναμικές Γλώσσες}

Ο όρος είναι λίγο ασαφής αλλά γενικά ως δυναμική γλώσσα εννοούμε μια γλώσσα
προγραμματισμού πολύ υψηλού επιπέδου, που παρουσιάζει συμπεριφορές υψηλής
αφαιρετικότητας κατά της εκτέλεση του προγράμματος, σε αντίθεση με άλλες γλώσσες
στις οποίες αυτό λαμβάνει χώρα κατά την μετάφραση του κώδικα σε κώδικα μηχανής.
Συνήθως τα προγράμματα αυτών των γλωσσών δεν μεταφράζονται απευθείας, αλλά ένα
ειδικό πρόγραμμα – το οποίο καλείται μεταγλωττιστής (interpreter) αναλαμβάνει να
τα "τρέξει", με την όλη διαδικασία της μετατροπής (του υψηλού επιπέδου κώδικα σε
κώδικα μηχανής) να λαμβάνει χώρα κατά το runtime· δηλαδή κατά την διάρκεια που ο
χρήστης τρέχει το πρόγραμμα και όχι κατά την μεταγλώττιση όπως συμβαίνει με
άλλες γλώσσες εξίσου υψηλού επιπέδου (βλ. Rust).

Τα χαρακτηριστικά και οι υποκατηγορίες των δυναμικών γλωσσών βρίθουν και η
ολοκληρωμένη λεπτομερής απαρίθμησή τους είναι εκτός των σκοπών αυτής της
εργασίας. Το σημαντικότερο κοινό χαρακτηριστικό είναι η χρήση του μεταγλωττιστή
και το "τρέξιμο" του προγράμματος στο περιβάλλον που δημιουργεί αυτό. Η
λειτουργία δηλαδή αυτή είναι σαν μια εικονική μηχανή και αυτό μας δίνει
απευθείας την δυνατότητα για ένα ακόμα επίπεδο αφαιρετικότητας στα design
patterns του προγραμματισμού μας. Έτσι έχουμε πράγματα όπως metaprogramming και
φυσικά δυναμικούς τύπους.

Οι δυναμικοί τύποι είναι το σημαντικότερο χαρακτηριστικό από την μεριά του
χρήστη, φυσικά για την ευκολία που δίνει σε αυτόν η εκάστοτε γλώσσα. Ο χρήστης
δεν χρειάζεται να δηλώσει ρητά τον τύπο μιας μεταβλητής. Αυτος συμπεραίνεται από
την αρχικοποίηση ή τα "συμφραζόμενα" της μεταβλητής μέσα στο πρόγραμμα. Επίσης
σημαντικό είναι ότι σε μερικές από αυτές τις γλώσσες μπορεί να αλλάξει κατά την
διάρκεια της εκτέλεσης.

Άλλο ένα τέτοιο σύγχρονο χαρακτηριστικό υψηλού επιπέδου είναι το just-in-time
compilation αλλά δεν θα ασχοληθούμε καθόλου με αυτό.

Οι δυναμικές γλώσσες ήταν πάντα δημοφιλείς, αλλά στις μέρες μας οι καινούργιες
συνθήκες, η ολοένα αυξημένη υπολογιστική ισχύ, και οι μεγάλες ομάδες (με πληθώρα
αναγκών) πίσω από τον σχεδιασμό των γλωσσών, έχουν οδηγήσει σε νέες  πτυχές στον
κόσμο των γλωσσών και του προγραμματισμού. Πολύ συχνά "αναδύονται" καινούργια
χαρακτηριστικά για συγκεκριμένες ανάγκες ή κάποιο είδος  αφαίρετικότητας τα
καταστεί πιο εύκολα. Η ώθηση αυτή, που διέπει αυτά τα  communities των δυναμικών
γλωσσών, είναι μια ισορροπία μεταξύ πρακτικότητας και κομψότητας. Τα
χαρακτηριστικά των γλωσσών αυτών τείνουν να μεγαλώνουν (με  εξαίρεση την
Python), στην οποία ακόμα διατηρείται ένα μινιμαλιστικό mindset. Σχεδόν σε όλες,
αντί για κάποιο καινούργιο abstraction (το οποίο θα προερχόταν  ή θα οδηγούσε σε
κάποια καινοτομία), προτιμάται ένα μεγάλο πλήθος μικρών μικρών  βελτιωτικών
χαρακτηριστικών, καθώς στοχεύουν να είναι εύκολες στην καθημερινή  χρήση και από
τον πιο ανειδίκευτο προγραμματιστή. Θα μπορούσε κανείς να πει,  ότι αυτές οι
γλώσσες είναι περισσότερο βιβλιοθήκες (libraries) πάνω σε μια απλή  γλώσσα (core
language). Τέλος, είναι προφανές, ότι το λιγότερο σημαντικό  χαρακτηριστικό για
αυτές τις γλώσσες είναι οι επιδόσεις. Πολλές φορές γίνονται  επιλογές (κατά τον
σχεδιασμό τους) υπέρ της ευκολίας χρήσης αντί των επιδόσεων. Όμως ακόμα και σε
άλλη περίπτωση, λόγω του μεγάλου αριθμού constructions στις  γλώσσες, η στατική
ανάλυση, το inference και η βελτιστοποίηση έχουν καταστεί  εξαιρετικά δύσκολες.


\subsection{Python}

\subsubsection{Γενικά - Ιστορία}

Συγκεκριμένα η γλώσσα, με την οποία θα ασχοληθούμε και στην οποία θα
υλοποιήσουμε το module, είναι η \textit{Python}\cite{python}. Η Python είναι μια
γενικού σκοπού δυναμική, "πολύ-παραδειγματική", υψηλού επιπέδου γλώσσα η οποία
είναι εξαιρετικά δημοφιλής εδώ και πολλά χρόνια. Η φιλοσοφία της δίνει βάση στην
καλή αναγνωσιμότητα του κώδικάς της και στην ευκολία της χρήσης. Υπάρχουν
υλοποιήσεις σχεδόν σε όλες τις  πλατφόρμες\footnote{π.χ. σε C, C\#, Java κλπ, με
την πιο δημοφιλής (και το reference για τις άλλες) να είναι η λεγόμενη CPython
σε C} και πολλές διαφορετικές εκδόσεις\footnote{βλ. Stackless Python} της και
πειράματα\footnote{βλ. pypy}. Επιπλέον λόγω της δημοτικότητάς της έχει επηρεάσει
πολλές άλλες γλώσσες όλων των ειδών και έχει ουσιαστικά συμβάλει στην σημερινή
εικόνα του κόσμου των υπολογιστών.

Βασίζεται στην φιλοσοφία των πρωταρχικών ώριμων γλωσσών (όπως C, Java), δηλαδή
το κυρίως \textit{προγραμματιστικό παράδειγμά} της είναι ο "
Προστακτικός-Διαδικαστικός Προγραμματισμός" (imperative/declarative
programming) με πολλά στοιχεία – και ένα καλό σύστημα αντικειμένων
(object-oriented programming). Επίσης λέμε ότι "ξεχωρίζει" τις έννοιες
δεδομένων και κώδικα – δηλαδή ακολουθεί το παράδειγμα της C και όχι της lisp.
Έχει παρόλα αυτά και πολλές επιρροές και χαρακτηριστικά από συναρτησιακό
προγραμματισμό (functional programming).

Σε αντίθεση με άλλες παρόμοιες γλώσσες, η Python προτιμά τον μινιμαλισμό. Οι
τελικές – σχετικά αυστηρές – αποφάσεις στον σχεδιασμό της λαμβάνονται από τον
Guido van Rossum, ο οποίος έχει περιπαικτικά τον τίτλο του "Benevolent Dictator
For Life".\cite{guido}

\subsubsection{Χαρακτηριστικά - Ιδιαιτερότητες}

Από την άλλη, όπως οι περισσότερες γλώσσες, βασικό χαρακτηριστικό που τις διέπει
είναι η παντελής έλλειψη δηλώσεων (\textit{no declaration notion}). Κάθε
πρόγραμμα χτίζεται με εντολές. Υπάρχουν κάποιες εκφράσεις που μπορεί να μοιάζουν
με δηλώσεις, όπως οι "δήλωση" συνάρτησης. Στην πραγματικότητα δεν είναι, και
απλώς δημιουργείται ένα αντικείμενο (runtime object) που δρα ως συνάρτηση.
Όμοιως και στην περίπτωση των κλάσεων, και των module\footnote{βλ.
\texttt{import} statement}, τα οποία αποτελούν βασικό κομμάτι του οικοσυστήματος
της γλώσσας. Πράγματι, \textit{τα πάντα} είναι ένα runtime αντικείμενο για την
Python. Άπαξ και δημιουργηθεί ένα αντικείμενο, μια αναφορά (reference) σε αυτό,
θα αποθηκευτεί στην τοπική λίστα ονομάτων (\textit{namespace}), η οποία στην
Python λέγεται \textit{module}. Αυτή είναι και η "μονάδα" του προγράμματος στην
Python. Το κατώτερο module στην ιεραρχία λέγεται φυσικά main module και είναι η
αντίστοιχη main συνάρτηση του κάθε προγράμματος.

Άλλα χαρακτηριστικά της γλώσσας περιλαμβάνουν εξαιρετικά καλό σύστημα γεννητόρων
(\textit{generators}) και επαναληπτών (\textit{iterators}) που διευκολύνουν σε
μεγάλο βαθμό κάποιες συγκεκριμένες περιπτώσεις, σύστημα metaprogramming
βασισμένο σε μετακλάσεις, σύστημα σχολίων χτισμένο μέσα στη γλώσσα (docstrings),
κ.α.

Ακολουθεί ένα μικρό παράδειγμα για μια πρώτη γεύση με την γλώσσα. Είναι κομμάτι
της build-in βιβλιοθήκης και υλοποιεί έναν μετρητή με την μορφή iterator.
Σημαντική είναι η χρήση whitespace για το identation. Αυτό οδηγεί σε κώδικα με
ακόμα μεγαλύτερη αναγνωσιμότητα.
\lstinputlisting[language=Python]{example.py}

%------------------------------------------------------------------------------

\section{Μεταγλώττιση Δυναμικών Γλωσσών}

Ο λεγόμενος μεταγλωττιστής (interpreter) είναι ο αντίστοιχος του μεταφραστής
(compiler) στις "στατικές" γλώσσες. Αντί να μεταφράζει εξ ολοκλήρου το
πρόγραμμα σε κώδικα μηχανής (και να εκτελεί όλες τις διαδικασίες για τις οποίες
είναι προγραμματισμένος) κατά την διάρκεια του compile-time (την ώρα που τρέχει
ο compiler για να "παράγει" το πρόγραμμα), το κάνει όταν αυτό χρειαστεί - και
αν χρειαστεί - κατά την διάρκεια του runtime. Όπως είδαμε στην προηγούμενη
ενότητα, δεν υπάρχει μια σταθερή δομή δηλώσεων (η οποία να μπορεί να αναλυθεί),
οπότε η μεταγλώττιση και ειδικά η βελτιστοποίηση είναι πολύ δύσκολες. Θεωρητικά
η γλώσσα θα μπορούσε να δημιουργήσει μια κλάση με εκατοντάδες τελείως
διαφορετικούς τρόπους βάσει αποτελεσμάτων από NP-complete υπολογισμούς και
εξωτερικούς παράγοντες.\cite{rigo2005}

\subsection{pypy}

\subsubsection{Τι είναι το pypy}

Το framework που θα χρησιμοποιήσουμε λέγεται pypy\cite{pypy}. Το pypy ξεκίνησε
το 2007 ως μια απόπειρα για έναν interpreter της Python γραμμένο στην ίδια την
γλώσσα\footnote{βλ. bootstrapping}. Από τότε έχει εξελιχθεί σε ένα ολοκληρωμένο
\textit{framework} με πολλές μοντέρνες δυνατότητες όπως \textit{JIT
compilation}. Αποτελείται από 2 μεγάλα subprojects:

\begin{itemize}

\item \textbf{RPython framework}

Το πρώτο είναι ένα compiler framework. Είναι γραμμένο σε κανονική Python, και
ουσιαστικά είναι μια βάση – ένα πρόγραμμα το οποίο μπορεί να παράγει compilers.
Μπορεί να "διαβάσει" μια ειδική περιορισμένη έκδοση της Python (που λέγεται
RPython). Οι διαφορές που έχει είναι ότι στερείται κάποιων κατασκευών υψηλής
αφαιρετικότητας. Έχει παρόλα αυτά πολλά από τα γνωστά χαρακτηριστικά της Python.

Σε αυτό το framework έχουν γραφτεί πλέον μεταφραστές και σε άλλες γλώσσες πέραν
μόνο της αρχικής Python, που "περιλαμβάνεται" στο project. Το πιο γνωστό
παράδειγμα είναι το Topaz\cite{topaz}.


\item \textbf{pypy}

Το δεύτερο πρότζεκτ είναι το pypy. Ένας compiler της Python γραμμένος βάσει του
προηγούμενου framework, σε RPython. Έχει πλέον ξεπεράσει σε ταχύτητα και
αποδοτικότητα τον CPython, τον reference compiler της γλώσσας. Οι λόγοι που δεν
ανακηρύσσεται αυτός reference compiler είναι "διοικητικοί"\footnote{βλ. Guido
van Rossum}. Βέβαια υπάρχουν και μερικά προβλήματα με παλιό κώδικα (legacy code)
και library support. Οι λόγοι μάλλον που ο compiler δεν είναι drop-in
replacement βέβαια, είναι λόγοι εμπιστοσύνης.

\end{itemize}

Φυσικά όπως όλα τα μεγάλα project έχει modular δομή κώδικα για πιο εύκολη
διαχείριση του τεράστιου πλέον όγκου του. Εμείς θα υλοποιήσουμε ένα τέτοιο
module στο υποσύστημα του backend optimization.

\subsubsection{Ανάλυση προγραμμάτων}

Το pypy, όπως και όλοι οι μεταγλωττιστές δυναμικών γλωσσών, αναλύει το κάθε
πρόγραμμα "ζωντανά" (live program analysis), δηλαδή στην μνήμη και όχι ως
"νεκρά"\footnote{στατικά} αρχεία. Αυτό σημαίνει ότι το πρόγραμμα φορτώνεται
(ίσως και ολόκληρο) αρχικά στην μνήμη και προχωρά η διαδικασία της
μεταγλώττισης. Όταν φτάσει σε ένα αρκετά καλό σημείο, θα μειωθεί η
"δυναμικότητά" του και θα αναλυθούν τα αντικείμενα που έχουν προκύψει στην
μνήμη. Ουστιαστικά η RPython είναι το υποσύνολο της Python που περιλαμβάνει μόνο
όσα χαρακτηριστικά υποστηρίζονται από αυτό το σύστημα ανάλυσης.

Στην περίπτωση που η ανάλυση γινόταν βάσει των στατικών αρχείων, τότε ουσιαστικά
θα ήταν σαν να "ακυρώναμε" την έννοια της δυναμικότητας της γλώσσας. Από την
άλλη, και η ανάλυση μιας \textit{σταθερής} εικόνας του προγράμματος στην μνήμη,
θα ήταν το ίδιο. Για αυτό και το σύστημα του pypy είναι εξαιρετικά δυναμικό.
Μπορεί να χειριστεί ακόμα και τελείως δυναμικά κομμάτια κώδικα εφόσον η είσοδος
της ροής σε αυτά είναι οριοθετημένη (bounded)\cite{rigo2005}. Είναι σημαντικό να
ξανά-αναφέρουμε ότι και το ίδιο το pypy είναι γραμμένο σε RPython, και αυτό
σημαίνει ότι και το ίδιο υφίσταται την ίδια δυναμική ανάλυση και απολαμβάνει τα
ίδια πλεονεκτήματα αυτής.

%------------------------------------------------------------------------------

%------------------------------------------------------------------------------

\chapter{Βελτιστοποίηση δυναμικού κώδικα}
\label{chapter2}

\section{Γενικά}

Η βελτιστοποίηση είναι η διαδικασία κατά την οποία ένα κομμάτι κώδικα
τροποποιείται έτσι ώστε να καταστεί πιο αποτελεσματικό - είτε από άποψη χώρου
είτε από άποψη χρόνου - χωρίς να αλλάξουν τα αποτελέσματα που δίνει ή τα side-
effects που προκαλεί. Ο χρήστης δεν θα πρέπει να αντιληφθεί την αλλαγή και απλά
να δει το πρόγραμμα του να τρέχει πιο γρήγορα και/ή να απαιτεί μικρότερη μνήμη.
Η βελτιστοποίηση κώδικα είναι το κεντρικό θέμα αυτής της εργασίας. Πάνω σε αυτό
το θέμα γίνεται σήμερα το μεγαλύτερο κομμάτι ακαδημαϊκής έρευνας από τα
πανεπιστήμια του κόσμου, καθώς τα υπόλοιπα κομμάτια ενός
μεταφραστή/μεταγλωττιστή θεωρούνται τετριμμένα. Η θεωρία των parsers και των
semitics analyzers για παράδειγμα έχει φτάσει σε ένα τυπικό σημείο κορεσμού, και
η αντίληψη που έχουμε για αυτά είναι σχεδόν ολοκληρωμένη. Από την άλλη η
βελτιστοποίηση κώδικα, παρά το γεγονός ότι μελετάται από τις απαρχές της εποχής
των ηλεκτρονικών υπολογιστών, είναι ακόμα σε πρώιμα στάδια.  Είναι δεδομένο ότι
οι compilers για τις "ώριμες" γλώσσες θα παράγουν σωστό κώδικα αλλά θα κριθούν
σε σχέση με τις άλλες για το πόσο καλό βελτιστοποιημένο κώδικα παράγουν.

Δηλαδή ο κάθε οργανισμός/πανεπιστήμιο/μεταγλωττιστής εκτελεί την βελτιστοποίηση
διαφορετικά. Αυτό γιατί τα περισσότερα προβλήματα βελτιστοποίησης είναι NP-
complete οπότε η πλειοψηφία των αλγόριθμων θα πρέπει να βασιστεί σε ευρετικά και
σε προσεγγίσεις. Είναι πολλές φορές δυνατόν ο ένας αλγόριθμος να παράγει σωστό
και γρήγορο κώδικα σε μια περίπτωση προβλήματος ενώ σε μια παρόμοια, όχι απλά να
μην υπάρχει βελτίωση, αλλά να υπάρχει και απότομη αύξηση χρόνου και/ή μνήμης.
Παρόλα αυτά οι περισσότεροι αλγόριθμοι τείνουν να δουλεύουν καλά για την
πλειοψηφία των προβλημάτων.

Φυσικά η καλύτερη - και έξυπνη - βελτιστοποίηση θα έρθει από την μεριά του
προγραμματιστή. Κανένας αλγόριθμος δεν είναι τόσο εξύπνος ακόμα ώστε να
αντικαταστήσει τον άνθρωπο π.χ. στην επιλογή του κατάλληλου αλγορίθμου για
ταξινόμηση. Με άλλα λόγια, όσων αφορά την ανάλυση πολυπλοκότητας σε big-O (τον
καλύτερο "κριτή" αλγορίθμων που έχουμε), οι αλγόριθμοι βελτιστοποίησης μπορούν
να κάνουν αλλαγές μόνο σε τάξη μεγέθους σταθεράς. Βέβαια αν ο αλγόριθμος είναι
ιδανικός από την πλευρά του "έξυπνου" χρήστη, τότε τέτοιες μικρές βελτιώσεις
μπορούν να κάνουν την διαφορά.

Από την άλλη ο προγραμματιστής θα πρέπει ιδανικά να περιορίζεται στις έξυπνες
επιλογές αλγορίθμου κλπ. και να μην επιχειρεί να βελτιώσει το πρόγραμμα πρόωρα·
κατά την πρώτη δηλαδή φάση της σχεδίασης και υλοποίησης, γιατί όποια τυχόν
βελτιστοποίηση από μέρους του μεταφραστή θα γίνει βάσει ιδιοματικής σύνταξης της
γλώσσας. Με άλλα λόγια, από ένα απλό loop αρχικοποίσης θα παραχθεί πολύ πιο
ποιοτικός κώδικας μηχανής εκ μέρους του μεταφραστή, σε σχέση με κάποιο απόκρυφα
δύσκολο τρόπο αρχικοποίησης, που ο προγραμματιστής επέλεξε επειδή θεώρησε ότι θα
είναι πιο γρήγορος. Στα λόγια του Donald Knuth: \textit{"Premature optimization
is the root of all evil in programming"}.\cite{knuth}

Η βελτιστοποίηση κώδικα δε θα πρέπει φυσικά σε καμία περίπτωση να πειράζει την
ορθότητα του προγράμματος. Θα πρέπει το παραγόμενο βελτιστοποιημένο πρόγραμμα να
αποδίδει τα ίδια αποτελέσματα στον τελικό χρήστη και να παράγει τις ίδιες
"παρενέργειες" (side-effects) σε όλες τις περιπτώσεις και για όλες τις εισόδους.
Σαφώς το πρόγραμμα θα πρέπει να είναι σωστό εκ μέρους του προγραμματιστή για να
μπορεί ο μεταφραστής να εγγυηθεί το παραπάνω.

Σημαντικός είναι ο καθορισμός του πότε είναι δυνατός και πότε απαιτείται η
βελτιστοποίηση. Φυσικά τον τελευταίο λόγο και σε αυτή την περίπτωση τον έχει ο
προγραμματιστής. Όλοι οι μοντέρνοι μεταφραστές έχουν την επιλογή πλήρους
απενεργοποίησης των διαδικασιών βελτιστοποιήσης ενώ πολλοί προσφέρουν και
δυνατότητες επιλογής συγκεκριμένων διαδικασιών (π.χ. μόνο αντικατάσταση
βαθμωτών).

Αξίζει να αναφέρουμε ότι η βελτιστοποίσης βασίζεται σε πολύ μεγάλο βαθμό στην
αρχιτεκτονική του κάθε υπολογιστή και το αποτέλεσμα μπορεί να είναι δραστικά
διαφορετικό όταν μεταφράσουμε για κάποια άλλη αρχιτεκτονική. Οι ώριμοι
μεταφραστές είναι ικανοί να μεταφράσουν για όλες τις σύγχρονες αρχιτεκτονικές.

Χονδρικά η διαδικασία μεταγλώττισης έχει ως εξής: λεξική ανάλυση (lexical
analysis), συντακτική ανάλυση (syntactic analysis), σημαντική ανάλυση
(semantics) όπου αντιστοιχίζονται τα τυχόν token με "ιδιότητες" της γλώσσας
(μέχρι εδώ αντιλαμβάνεται όλα τα συντακτικά λάθη), και έπειτα παραγωγή
intermediate κώδικα. Σε αυτό το σημείο ο κώδικας αυτός θα υποστεί ανάλυση (βλ.
παρακάτω) και θα βελτιωθεί.

%------------------------------------------------------------------------------

\section{Τεχνικές Βελτιστοποίσης}

Οι περισσότερες τεχνικές βελτιστοποίησης βασίζονται σε ανάλυση του
\textit{"control flow"} (ροή) του προγράμματος κατά το runtime (Control flow
analysis) Μέχρι και πριν από το σημείο της ανάλυσης αυτής ο μεταφραστής έχει
μόνο τυπικές πληροφορίες για το τι κάνει το πρόγραμμα που μεταφράζει. Εδώ
βρίσκει περισσότερες "σημαντικές" πληροφορίες για την φύση του προγράμματος με
το να αναλύει την ροή του (δηλαδή τα forks λόγω if, τα loops κ.α.). Φυσικά ποτέ
δεν έχει πλήρη επίγνωση.

Η ανάλυση αυτή γίνεται με το να κατασκευάζεται το \textit{γράφημα ροής} (control
flow graph), το οποίο καταγράφει όλη την ροή δηλαδή όλα τα πιθανά "μονοπάτια"
που είναι δυνατόν να ακολουθήσει το πρόγραμμα. Το βασικό στοιχείο είναι η
\textit{συνάρτηση}. Δημιουργείται ένα γράφημα δηλαδή για την κάθε συνάρτηση με
ένα σημείο εισόδου (entry point), στην αρχή φυσικά της συνάρτησης και ένα ή
περισσότερα σημεία εξόδου ανάλογα με τις ανάγκες. Για την παραγωγή του κάθε
γραφήματος ο κώδικας χωρίζεται σε κομμάτια (\textit{basic blocks}), στα οποία η
ροή του προγράμματος ξεκινάει μόνο από ένα σημείο στην αρχή και τελειώνει σε ένα
και μοναδικό σημείο στο τέλος. Τα τυχόν παρακλάδια (branches ή forks) του
προγράμματος οργανώνονται βάσει τον blocks αυτών. Αυτό σημαίνει ότι δεν μπορούν
να υπάρξουν branches στη μέση των blocks, οπότε όλες οι δηλώσεις και οι εντολές
(\textit{statements}) θα πρέπει να τρέξουν διαδοχικά. Τα branches όπως είπαμε
βρίσκονται έξω από τα blocks και οδηγούν στα επόμενα σύμφωνα με την ροή του
προγράμματός μας. Από το σημείο του γραφήματος, μόνο το πρώτο statement είναι
"ορατό" και αν η ροή πάει στο block, θα πρέπει αυτό να τρέξει ολόκληρο.

Ένα block μπορεί να ξεκινάει είτε με το entry point ενός branch είτε να είναι το
target ενός branch, και να τελειώνει είτε με μια οδηγία άλματος (jump statement)
είτε με οδηγία συνθήκης (conditional statement· δηλαδή if) είτε τέλος με οδηγία
"επιστροφής" (return statement), η οποία τερματίζει ολόκληρο το graph της
εκάστοτε συνάρτησης.

Οι μέθοδοι βελτιστοποίσης χωρίζονται σε τοπικές (local) και καθολικές (global).
Οι πρώτες δουλεύουν αποκλειστικά αλλάζοντας statements κ.α. μέσα στα blocks και
οι δεύτερες δουλεύουν στις σχέσεις μεταξύ των blocks. Φυσικά πολύ πιο απλές στον
σχεδιασμό και στην υλοποίηση είναι οι πρώτες. Παρακάτω δίνονται κάποια
παραδείγματα. (Πολλές από τις τοπικές που θα συζητήσουμε πρώτα έχουν αντίστοιχες
καθολίκες που βασίζονται στην ίδια αρχή.) Επίσης όπως θα δούμε  υπάρχουν οι
λεγόμενες βελτιστοποιήσεις μηχανής.


\subsection{Τοπικές Μέθοδοι Βελτιστοποίησης}

\begin{itemize} \item Αναδίπλωση Σταθερών

Με τον όρο "Αναδίπλωση Σταθερών" (\textit{Constant folding}) αναφερόμαστε στον
εντοπισμό των τελεστέων κάποιων δηλώσεων/διαδικασιών/οδηγιών που μπορούν να
αποτελέσουν ή ήδη αποτελούν σταθερές. Στην πιο απλή μορφή της θα αντικαταστήσει
αριθμιτικές πράξεις, π.χ.: Ένα statement όπως: $ a = 4 * 2 + 3 $ θα μετατραπεί
σε $ a = 11 $, αποφεύγοντας έτσι τις πράξεις αύτες κατά το runtime. Η απόδειξη
ότι αυτή η αντικατάσταση μπορεί να γίνει διατηρώντας την ορθότητα είναι
προφανής, αφού η πράξη αυτή απλώς πρέπει να εκτελεστεί μια φορά. Άπαξ και την
εκτελέσει ο μεταφραστής, η ορθότητα διατηρείται.

\item Αναδίπλωση Επαναλήψεων

Σε πολλές περιπτώσεις – ειδικά αν ο μεταγλωττιστής αναγνωρίσει ότι πρόκειται για
επανάληψη με συγκεκριμένο σκοπό π.χ. αρχικοποίηση πίνακα – η επανάληψη μπορεί να
εξαλειφθεί. Στην θέση της θα εισαχθεί ίσως κάποιο ειδικό construction του
compiler ή της γλώσσας για αρχικοποίηση σε $O(1)$ (ή έστω κάτι πιο γρήγορο από
το αρχικό).

\item Αναδιάδοση Σταθερών

Σε περίπτωση που ένα statement που έχει μορφή σταθεράς (είτε όπως προηγουμένως 
μια σειρά από πράξεις είτε απλώς ένας αριθμός είτε τέλος μια συμβολοσειρά) 
ανατεθεί σε μια μεταβλητή, τότε ο μεταφραστής μπορεί να την αντικαταστήσει με 
την ίδια την σταθερά. Αυτό μπορεί να συμβεί μόνο στην περίπτωση που ενδιάμεσα 
στις χρήσεις της μεταβλητής δεν υπάρχει άλλη ανάθεση (δηλαδή αλλαγή των 
περιεχομένων της μεταβλητής). Μπορεί αρχικά να φαίνεται ότι η βελτίωση θα είναι 
μικρή αφού σώνουμε έναν μικρό αριθμό προσπελάσεων μνήμης αλλά σε κάποιες 
αρχιτεκτονικές (π.χ. RISC) υπάρχουν τεράστια αποτελέσματα και η μέθοδος αυτή 
είναι εξαιρετικά αποτελεσματική καθώς μειώνεται και ο αριθμός των καταχωρητών (
registers) που χρησιμοποιούνται στην τελική έκδοση του προγράμματος σε κώδικα 
μηχανής. Αυτό συμβαίνει γιατί στην περίπτωση της MISC το σύστημα 
διευθυνσιοδότησης δουλεύει με πρόσθεση ένος register και ενός σταθερού offset.

\item Αναδιάδοση

Ομοίως με παραπάνω μπορεί να γίνει κάτι παρόμοιο στην περίπτωση απλώς
μεταβλητών.

\item Αλγεβρική Απλοποίηση

Ο μεταφραστής μπορεί να χρησιμοποιήσει συγκεκριμένους αλγεβρικούς συνδιασμούς (
τους οποίους γνωρίζει \textit{a-priori}) και ξέρει ότι λειτουργούν και τρέχουν 
πιο γρήγορα από άλλους. Επίσης είναι δυνατόν να χρησιμοποιηθούν αλγεβρικές 
ιδιότητες για να απλοποιηθούν οι παραστάσεις και να χρειάζονται λιγότερες 
πράξεις, λ.χ:
\[
(x+y)^2 = x^2 + 2xy + y^2
\]
Προφανώς το αριστερό μέλος θέλει πολύ λιγότερες πράξεις από το δεξί.
Τέλος πολλές πράξεις μπορούν να αφαιρεθούν αν κριθούν μη-απαραίτητες. Λαμπρό 
παράδειγμα η πρόσθεση με το μηδέν κλπ.

\item Ενίσχυση Δύναμης Τελεστή

Κατά την εφαρμογή της μεθόδου "Ενίσχυσης Δύναμης Τελεστή" ο μεταφραστής
αποπειράται να αντικαταστήσει συγκεκριμένους τελεστές με άλλους λιγότερο
"ακριβούς". Φυσικά αυτό έχει να κάνει με την αρχιτεκτονική. Λ.χ σε κάποια μπορεί
η έκφραση $ x * 2 $ να είναι πιο ακριβή από την $ 2 * x $.

\item Διαγραφή "Νεκρού" Κώδικα

Στην περίπτωση που ο μεταφραστής αντιληφθεί ότι κάποια σημεία του γραφήματος
ροής δεν μπορούν να προσπελαστούν σε καμιά περίπτωση και με καμία είσοδο, τότε
το απροσπέλαστο αυτό block καλείται \textit{νεκρός κώδικας} και μπορεί να
διαγραφεί με ασφάλεια.

\item Εξάλειψη Εκφράσεων Κοινών Αποτελεσμάτων

Όταν ο μεταφραστής εντοπίσει δύο εκφράσεις που παράγουν το ίδιο αποτέλεσμα τότε 
μπορεί με ασφάλεια να αφαιρέσει την μια και να εισάγει ένα reference προς την 
άλλη.

\end{itemize}

\subsection{Καθολικές Μέθοδοι Βελτιστοποίησης}

Μέθοδοι που αλλάζουν statements και συνδέσεις (links) μεταξύ των blocks.

\begin{itemize}

\item Μετακίνηση Κώδικα (Code Motion)

Σε κάποιες περιπτώσεις ο μεταφραστής εντοπίζει κομμάτια κώδικα μέσα σε διάφορα
και/ή ξεχωριστά block, που είτε επαναλαμβάνονται είτε παράγουν το ίδιο
αποτέλεσμα. Τότε αυτά μπορούν να μετακινηθούν σε ένα σημείο και να συμβούν μια
φορά. Σημαντικότερο παράδειγμα, κώδικας μέσα σε loop που απαιτείται να τρέξει
μόνο μια φορά και όχι σε κάθε επανάληψη (iteration). [Λέγεται επίσης και code
hoisting.]

\end{itemize}

\subsection{Μέθοδοι Βελτιστοποίησης Μηχανής}

\begin{itemize}

\item Δέσμευση Καταχωρητών

Ίσως η πιο σημαντική μέθοδος βελτιστοποίησης είναι η σωστή δέσμευση των
καταχωρητών (registers) του επεξεργαστή. Οι καταχωρητές είναι η πιο γρήγορη
μορφή μνήμης αφού βρίσκονται πολύ κοντά στον επεξεργαστή. Για αυτό όμως είναι
και λίγοι και σπανίζουν. Οι πιο αποτελεσματικές μέθοδοι πρέπει να προσδιορίσουν
ποιές μεταβλητές και πότε θα βρίσκονται στους καταχωρητές για να
ελαχιστοποιήσουν τα memory accesses, τα conflicts και τα races. Δηλαδή να
ελαχιστοποιήσουν την κίνηση δεδομένων από και προς τον επεξεργαστή. Ένας πολύ
αποτελεσματικός αλγόριθμος είναι ο λεγόμενος "χρωματισμός" (register  coloring).
Εκτός από τους καταχωρητές ο εκάστοτε αλγόριθμος θα πρέπει να  οργανώσει φυσικά
και όλη την ιεραρχία μνήμης, λαμβάνοντας υπόψιν του και τα  κρυφά επίπεδα μνήμης
της κάθε αρχιτεκτονικής (cache).

\item Χρονολόγηση Εντολών (Instruction Scheduling)

Άλλη μία σημαντική μέθοδος. Ο μεταφραστής καλείται να ανακαλύψει την 
ιδανικότερη σειρά με την οποία θα οργανωθούν οι εντολές στον χρόνο, έχοντας 
υπόψιν του τις ιδιορρυθμίες και τα ειδικά χαρακτηριστικά του κάθε επεξεργαστή 
και αρχιτεκτονικής. Οι λεπτομέρειες περιλαμβάνουν: ικανότητες pipelining, 
πλήθος διαθέσιμων εντολών (RISC/MISC), την χρήση big ή small endian κ.α.

\item Βελτιστοποιήσεις "Κλειδαρότρυπας" (Peephole)

Εδώ περιλαμβάνονται μέθοδοι που σχετίζονται με την εκάστοτε μηχανή. Σε κάθε 
επανάληψη εξετάζονται μερικές από τις επόμενες εντολές (εξού και το όνομα \
textit{κλειδαρότρυπα}) και βελτιώνονται αντίστοιχα. Παράδειγμα: η αποφυγή 
φόρτωσης δεδομένων όταν η προηγούμενη εντολή φορτώνει τα δεδομένα αυτόματα.

\end{itemize}

Στο επόμενο κεφάλαιο αναλύεται αρχικά η διακασία μεταγλώττισης του pypy (για
λόγους κατανόησης), και έπειτα πιο λεπτομερώς η μέθοδος που θα
χρησιμοποιήσουμε στην υλοποίηση του pypy module.

%------------------------------------------------------------------------------



\section*{Κώδικας}
\lstinputlisting[language=Python]{partial_escape.py}


%-----------------------------------------------------------
% BIBLIOGRAPHY
\printbibliography
%%%%%%%%%%%%%%%%%%%%%%%%%%%%%%%%%%%%%%%%%%%%%%%%%%%%%%%%%%%%
%	THE END
%%%%%%%%%%%%%%%%%%%%%%%%%%%%%%%%%%%%%%%%%%%%%%%%%%%%%%%%%%%%
\end{document}  